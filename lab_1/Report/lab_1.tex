\documentclass[a4paper, 14pt]{article}
\usepackage[T1,T2A]{fontenc}
\usepackage[utf8x]{inputenc}
\usepackage[english,russian]{babel}
\usepackage{listings}
\usepackage[russian]{babel}
\usepackage{graphicx}
\usepackage{listings}
\usepackage{color}
\usepackage{amsmath}
\usepackage{pgfplots}
\usepackage{url}
\usepackage{tikz}
\usetikzlibrary{calc,trees,positioning,arrows,chains,shapes.geometric,%
    decorations.pathreplacing,decorations.pathmorphing,shapes,%
    matrix,shapes.symbols}

\usepackage{titlesec}
\titleformat*{\section}{\LARGE\bfseries\centering}
\titleformat*{\subsection}{\Large\bfseries\centering}
\titleformat*{\subsubsection}{\large\bfseries}
\titleformat*{\paragraph}{\large\bfseries}
\titleformat*{\subparagraph}{\large\bfseries}
\lstset{ %
language=python,                 % выбор языка для подсветки (здесь это С)
basicstyle=\small\sffamily, % размер и начертание шрифта для подсветки кода
numbers=left,               % где поставить нумерацию строк (слева\справа)
numberstyle=\tiny,           % размер шрифта для номеров строк
stepnumber=1,                   % размер шага между двумя номерами строк
numbersep=5pt,                % как далеко отстоят номера строк от подсвечиваемого кода
showspaces=false,            % показывать или нет пробелы специальными отступами
showstringspaces=false,      % показывать или нет пробелы в строках
showtabs=false,             % показывать или нет табуляцию в строках
frame=single,              % рисовать рамку вокруг кода
tabsize=2,                 % размер табуляции по умолчанию равен 2 пробелам
captionpos=t,              % позиция заголовка вверху [t] или внизу [b] 
breaklines=true,           % автоматически переносить строки (да\нет)
breakatwhitespace=false, % переносить строки только если есть пробел
escapeinside={\#*}{*)}   % если нужно добавить комментарии в коде
}

\usepackage{geometry}
\geometry{left=2cm}
\geometry{right=1.5cm}
\geometry{top=1cm}
\geometry{bottom=2cm}

\usepackage{pgfplots}
\usepackage{filecontents}
\usetikzlibrary{datavisualization}
\usetikzlibrary{datavisualization.formats.functions}
\usepackage{graphicx}
\begin{filecontents}{lev_tab.dat}
2 1.24e-05
3 1.86e-05
4 3.22e-05
5 4.20e-05
6 6.38e-05
7 8.64e-05
\end{filecontents}

\begin{filecontents}{dam_lev_tab.dat}
2 1.43e-05
3 2.29e-05
4 3.37e-05
5 4.94e-05
6 7.42e-05
7 1.01e-04 
\end{filecontents}

\begin{filecontents}{dam_lev_rec.dat}
2 1.83e-05
3 9.10e-05
4 4.95e-04
5 2.75e-03
6 1.60e-02
7 8.83e-02
\end{filecontents}

\begin{document}

    \begin{titlepage}

        \begin{center}
            \large
            Государственное образовательное учреждение высшего профессионального образования\\
            “Московский государственный технический университет имени Н.Э.Баумана”
            \vspace{3cm}
            
            \textsc{Дисциплина: Анализ алгоритмов}
            \vspace{0.5cm}
                
            \textsc{Лабораторная работа №1}
            \vspace{1.5cm}
            
            {\LARGE ТЕМА-расстояния Левенштейна и Дамерау-Левенштейна.\\}
            \vspace{1.5cm}
            Студент группы ИУ7-55,\\   
            Аминов Тимур Саидович
            \vfill
            
            2019 г.
            
            \end{center}

    \end{titlepage}
\tableofcontents
	
	\newpage
	
        \section*{Введение}
        \addcontentsline{toc}{section}{Введение}
        \begin{flushleft}
        \parindent=1cm
        \textbf{Расстояние Левенштейна} - минимальное количество операций вставки одного символа, удаления одного символа и замены одного символа на другой, необходимых для превращения одной строки в другую.
        
        Расстояние Левенштейна используют:
        \begin{itemize}
        	\item для исправления ошибок в слове;
			\item для сравнения текстовых файлов утилитой diff и ей подобными;
			\item для сравнения ДНК в биоинформатике.
		\end{itemize}

        Цель работы: изучение метода динамического программирования на материале алгоритмов	Левенштейна и Дамерау-Левенштейна.
		Задачи работы:\\
		\begin{enumerate}
			\item Изучение алгоритмов Левенштейна и Дамерау-Левенштейна нахождения расстояния между
		строками;

			\item Применение метода динамического программирования для матричной реализации указанных
		алгоритмов;

			\item Получение практических навыков реализации указанных алгоритмов: двух алгоритмов в
		матричной версии и одного из алгоритмов в рекурсивной версии;

			\item Сравнительный анализ линейной и рекурсивной реализаций выбранного алгоритма
		определения расстояния между строками по затрачиваемым ресурсам (времени и памяти);

			\item Экспериментальное подтверждение различий во временнóй эффективности рекурсивной и
		нерекурсивной реализаций выбранного алгоритма определения расстояния между строками при
		помощи разработанного программного обеспечения на материале замеров процессорного времени
		выполнения реализации на варьирующихся длинах строк;

			\item Описание и обоснование полученных результатов в отчете о выполненной лабораторной
		работе, выполненного как расчётно-пояснительная записка к работе.
		\end{enumerate}
        \end{flushleft}

        \label{sec:intro}

    \begin{center}
    	\newpage
        \section{Аналитическая часть}
		\parindent=1cm
        \subsection{Описание алгоритмов}
        \begin{flushleft}
        Обозначим простейшие действия над двумя строками:
        \begin{enumerate}
        	\item D - удалить букву в одной из строк;
        	\item I - вставить букву в одной из строк;
        	\item R - заменить букву в одной из строк;
        	\item M - совпадение двух букв.
        \end{enumerate}
        \parindent=1cm
        Каждая операция имеет свою `цену` - у совпадения она 0, а у трех остальных - по 1.
        Задача нахождения расстояния Левенштейна между двумя строками сводится к поиску набора операций, которые нужно совершить, чтобы трансформировать одну строку в другую, причем их суммарная цена должна быть минимальной.
        
        Расстояние Дамерау-Левенштейна отличается от расстояния Левенштейна добавлением операции транспозиции - перестановки двух соседних символов(ее цена - 1).	
        
        Пусть $S_{1}$ и $S_{2}$ — две строки над некоторым алфавитом, тогда расстояние Левенштейна между ними можно подсчитать по следующей формуле (1):
        \begin{equation}
			D(i,j) = \left\{ \begin{array}{ll}
 			0, & \textrm{$i = 0, j = 0$}\\
 			i, & \textrm{$j = 0, i > 0$}\\
 			j, & \textrm{$i = 0, j > 0$}\\
			min(D1,D2,D3)
  			\end{array} \right.
		\end{equation}
		\\где:\\ 
		D1 = D(i, j-1)+1;\\
		D2 = D(i-1, j) + 1;\\
		D3 = D(i-1, j-1) + (0 если match($S_{1}[i]$, $S_{2}[j]$)), 1 иначе);\\
		match(a, b) = истина при а = b, ложь иначе;\\
		min(A1, A2, ..., AN) - минимум среди чисел A1, A2, ..., AN.\\
		
		В свою очередь, расстояние Дамерау-Левенштейна можно посчитать по формуле (2):
		\begin{equation}
			D(i,j) = \left\{ \begin{array}{ll}
			0, & \textrm{$i = 0, j = 0$}\\
			i, & \textrm{$j = 0, i > 0$}\\
 			j, & \textrm{$i = 0, j > 0$}\\
			min(D1,D2,D3,D4), & i>1 , j>1, a_{i} = b_{j-1},a_{i-1}=b_{j}  \\
			min(D1,D2,D3)
  			\end{array} \right.
		\end{equation}
		\\где:\\
		D4 = D(i-2, j-2) + 1, если i>1 ,j>1 и $a_{i} = b_{j-1},a_{i-1}=b_{j} $.\\
		\end{flushleft}
    \end{center}
    \begin{center}
    \subsection{Вывод}
    \end{center}

    В данном разделе были рассмотрены формулы для нахождения расстояния Левенштейна и Дамерау-Левенштейна, а также основные приминения этих расстояний.\\

    \begin{center}
    	\newpage
        \section{Конструкторская часть}
        \subsection{Разработка алгоритмов}
        \begin{figure}[h]
			\center{\includegraphics[width=1\linewidth]{lev_tab.png}}
			\caption{Схема алгоритма нахождения расстояния Левенштейна(итеративного)}
			\label{ris:image}
		\end{figure}
		\newpage
        \begin{figure}[h]
			\center{\includegraphics[width=1\linewidth]{dam_lev_tab.png}}
			\caption{Схема алгоритма нахождения расстояния Дамерау-Левенштейна(итеративного)}
			\label{ris:image}
		\end{figure}
		\newpage
        \begin{figure}[h]
			\center{\includegraphics[scale=0.42]{dam_lev_rec.png}}
			\caption{Схема алгоритма нахождения расстояния Дамерау-Левенштейна(рекурсивного)}
			\label{ris:image}
		\end{figure}
		\newpage
    \end{center}
    
    
    
\subsection{Вывод}
 	
    \begin{flushleft}	
	В данном разделе были рассмотрены схемы алгоритмов нахождения расстояния Левенштейна, Дамерау-Левенштейна, а также рекурсивный алгоритм нахождения расстояния Дамерау-Левенштейна.
    \end{flushleft}

    \begin{center}
    	\newpage
        \section{Технологическая часть}

        \subsection{Требования к программному обеспечению}
        \begin{flushleft}
        Входные данные: str1 - первое слово, str2 - второе слово.\\
		Выходные данные: значение расстояния между двух слов.
        \end{flushleft}
            \begin{figure}[h]
			\center{\includegraphics[width=1\linewidth]{idef0.png}}
			\caption{IDEF0-диаграмма, описывающая алгоритм нахождения расстояния Левенштейна}
			\label{ris:image}
		\end{figure}
		\subsection{Средства реализации}
		\begin{flushleft}
		В качестве языка был выбран python, для вычисления памяти был использован метод sys.getsizeof(), возвращающий память, затрачиваемую на объект. Для замеров времени была использована функция tick(), приведенная ниже:
		\begin{lstlisting}[label=some-code,caption=Функция замера времени]
		unsigned long long tick(void)
		{
    		unsigned long long d;
    		__asm__ __volatile__ ("rdtsc" : "=A" (d) );
    		return d;
		}
		\end{lstlisting}
		Для подключения ее в модуль python была создана библиотека libtick.so и подключена с помощью модуля ctypes.
		\end{flushleft}
		\newpage
        \subsection{Листинг кода}
        \begin{flushleft}
        \parindent=1cm
        \begin{lstlisting}[label=some-code,caption=Функция нахождения расстояния Левенштейна итеративно]
  		def lev_table(str_1, str_2):
			len_1 = len(str_1)
			len_2 = len(str_2)
			table = [[0 for j in range(len_2 + 1)] for i in range(len_1 + 1)]

			for i in range(len_1):
				table[i + 1][0] = table[i][0] + 1

			for i in range(len_2):
				table[0][i + 1] = table[0][i] + 1

			for i in range(1, len_1 + 1, 1):
				for j in range(1, len_2 + 1, 1):
					step_1 = table[i - 1][j] + 1
					step_2 = table[i][j - 1] + 1

					if str_1[i - 1] == str_2[j - 1]:
						diag = 0
					else:
						diag = 1

					step_3 = table[i - 1][j - 1] + diag

					table[i][j] = min(step_1, step_2, step_3)

			return table[len_1][len_2]
        \end{lstlisting}
        \begin{lstlisting}[label=some-code,caption=Функция нахождения расстояния Дамерау-Левенштейна итеративно]
		def dam_lev_table(str_1, str_2):
			len_1 = len(str_1)
			len_2 = len(str_2)
			table = [[0 for j in range(len_2 + 1)] for i in range(len_1 + 1)]

			for i in range(len_1):
				table[i + 1][0] = table[i][0] + 1

			for i in range(len_2):
				table[0][i + 1] = table[0][i] + 1

			for i in range(1, len_1 + 1, 1):
				for j in range(1, len_2 + 1, 1):
					step_1 = table[i - 1][j] + 1
					step_2 = table[i][j - 1] + 1

					if str_1[i - 1] == str_2[j - 1]:
						diag = 0
					else:
						diag = 1

					step_3 = table[i - 1][j - 1] + diag

					if i - 2 >= 0 and j - 2 >= 0 and str_1[i - 1] == str_2[j - 2] and str_1[i - 2] == str_2[j - 1]:
						step_4 = table[i - 2][j - 2] + 1
						table[i][j] = min(step_1, step_2, step_3, step_4)
					else:
						table[i][j] = min(step_1, step_2, step_3)

			return table[len_1][len_2]
        \end{lstlisting}
        \newpage
        \begin{lstlisting}[label=some-code,caption=Функция нахождения расстояния Дамерау-Левенштейна рекурсивно]
	def dam_lev_recursion(str_1, str_2):
			if len(str_1) == 0:
				return len(str_2)

			elif len(str_2) == 0:
				return len(str_1)

			else:
				if str_1[-1] == str_2[-1]:
					diag = 0
				else:
					diag = 1

				step_1 = dam_lev_recursion(str_1, str_2[:-1]) + 1
				step_2 = dam_lev_recursion(str_1[:-1], str_2) + 1
				step_3 = dam_lev_recursion(str_1[:-1], str_2[:-1]) + diag
				step_4 = dam_lev_recursion(str_1[:-2], str_2[:-2]) + 1

				if len(str_1) > 1 and len(str_2) > 1 and str_1[-1] == str_2[-2] and str_1[-2] == str_2[-1]:
					d = min(step_1, step_2, step_3, step_4)
				else:
					d = min(step_1, step_2, step_3)

				return d
        \end{lstlisting}
		\end{flushleft}
		
        \subsection{Тестирование}
        \begin{flushleft}
        \parindent=1cm
        Было организовано тестирование с помощью заранее подготовленных данных в виде пары строк и расстояния по Левенштейну и Дамерау-Левенштейну. Они приведены в таблице 1:\\
        	\begin{center}
  	Таблица 1. Тестовые данные для алгоритмов нахождения расстояния Левенштейна и Дамерау-Левенштейна.\\
	\end{center}
        
        \begin{tabular}{ | c | c | r | r | }
        \hline
s1 & s2 & р. Левенштейна & р. Дамерау-Левенштейна \\ \hline
'mod' & 'rot' & 2 & 2\\
'mod' & 'mdo' & 2 & 1\\
'mod' & 'ram' & 3 & 3\\
'mod' & 'omr' & 3 & 2\\
'modrotmod' & 'modrot' & 3 & 3\\
'' & 'modrot' & 6 & 6\\
'' & '' & 0 & 0\\
'' & 'a' & 1 & 1\\
'a' & 'b' & 1 & 1\\
'modr' & 'mord' & 2 & 1\\
\hline
        \end{tabular}
	
        Все реальные результаты совпали с ожидаемыми из таблицы 1, что подтвердило правильность работы программы.
                
        \end{flushleft}
    \end{center}
	\subsection{Сравнительный анализ потребляемой памяти}	
	Найдем память, затрачиваемую на объекты одного вызова каждой из функций:\\
	\begin{center}
  	Таблица 2. Память, потребляемая структурами в алгоритме Левенштейна\\
	\end{center}
	\scalebox{0.87}{	
	\begin{tabular}{|c|c|}
	\hline 
	Структура данных & Занимаемая память (байты) \\ 
	\hline 
	Матрица & 40 + 8 * [len(str1) + 1] + [len(str1) + 1] * [40 + 8 * (len(str 2) + 1)]\\ 
	\hline 
	2 строки & 2 * [49 + len(str)] \\ 
	\hline 
	6 вспомогательных переменных(int) & 168 \\ 
	\hline 
	2 счетчика (int) & 56 \\ 
	\hline 
	\end{tabular}
	}
	
	\begin{center}
  	Таблица 3. Память, потребляемая структурами в алгоритме Дамерау-Левенштейна\\
	\end{center}
	\scalebox{0.87}{	
	\begin{tabular}{|c|c|}
	\hline 
	Структура данных & Занимаемая память (байты) \\ 
	\hline 
	Матрица & 40 + 8 * [len(str1) + 1] + [len(str1) + 1] * [40 + 8 * (len(str 2) + 1)]\\ 
	\hline 
	2 строки & 2 * [49 + len(str)] \\ 
	\hline 
	7 вспомогательных переменных(int) & 196 \\ 
	\hline 
	2 счетчика (int) & 56 \\ 
	\hline 
	\end{tabular}
	}
	
			\begin{center}
  	Таблица 4. Память, потребляемая структурами в рекурсивном алгоритме нахождения расстояния Дамерау-Левенштейна за один вызов\\
	\end{center}
	
		\scalebox{0.87}{	
	\begin{tabular}{|c|c|}
	\hline 
	Структура данных & Занимаемая память (байты) \\ 
	\hline 
	4 вспомогательных переменных (int) & 112\\ 
	\hline 
	2 строки & 98 + len(str1) + len(str2) \\ 
	\hline 
	\end{tabular}
	}

		\begin{center}
	
	Максимальная глубина рекурсивного вызова функции - сумма длин двух слов. \\
	\end{center}

	\subsection{Оценка потребляемой памяти на 4 и 1000 символах}	
	Оценим алгоритмы на словах длинной 4 символа:\\

	\begin{center}
	Таблица 5. Память, потребляемая структурами в алгоритме нахождения расстояния Левенштейна\\
	\end{center}
	
	\scalebox{0.87}{	
	\begin{tabular}{|c|c|}
	\hline 
	Структура данных & Занимаемая память (байты) \\ 
	\hline 
	Матрица & 480\\ 
	\hline 
	2 строки & 106 \\ 
	\hline 
	6 вспомогательных переменных(int) & 168 \\ 
	\hline 
	2 счетчика (int) & 56 \\ 
	\hline 
	\textbf{Всего} & \textbf{808}\\
	\hline
	\end{tabular}
	}
	

	\begin{center}
  	Таблица 6. Память, потребляемая структурами в алгоритме нахождения расстояния Дамерау-Левенштейна\\
	\end{center}	
	\scalebox{0.87}{	
	\begin{tabular}{|c|c|}
	\hline 
	Структура данных & Занимаемая память (байты) \\ 
	\hline 
	Матрица & 480\\ 
	\hline 
	2 строки & 106 \\ 
	\hline 
	7 вспомогательных переменных(int) & 196 \\ 
	\hline 
	2 счетчика (int) & 56 \\ 
	\hline 
	\textbf{Всего} & \textbf{836}\\
	\hline
	\end{tabular}
	}	

	\begin{center}
  	Таблица 7. Память, потребляемая структурами в рекурсивном алгоритме нахождения расстояния Дамерау-Левенштейна\\
	\end{center}	
	\scalebox{0.87}{	
	\begin{tabular}{|c|c|}
	\hline 
	Структура данных & Занимаемая память (байты) \\ 
	\hline 
	4 вспомогательных переменных (int) & 140 * 8(максимальная глубина вызовов) = 896\\ 
	\hline 
	2 строки & 106 * 4/2(Усредненное значение) = 212 \\ 
	\hline 
	\textbf{Всего} & \textbf{1108}\\
	\hline	
	\end{tabular}
	}	

	
	Оценим алгоритмы на словах длинной 1000 символов:\\

	
		\begin{center}
	Таблица 8. Память, потребляемая структурами в алгоритме нахождения расстояния Левенштейна
	\end{center}

	\scalebox{0.87}{	
	\begin{tabular}{|c|c|}
	\hline 
	Структура данных & Занимаемая память (байты) \\ 
	\hline 
	Матрица & 8 064 096\\ 
	\hline 
	2 строки & 106 \\ 
	\hline 
	6 вспомогательных переменных(int) & 168 \\ 
	\hline 
	2 счетчика (int) & 56 \\ 
	\hline 
	\textbf{Всего} & \textbf{8064426}\\
	\hline
	\end{tabular}
	}


	\begin{center}
  	Таблица 9. Память, потребляемая структурами в алгоритме нахождения расстояния Дамерау-Левенштейна\\
	\end{center}
	\scalebox{0.87}{	
	\begin{tabular}{|c|c|}
	\hline 
	Структура данных & Занимаемая память (байты) \\ 
	\hline 
	Матрица & 8 064 096\\ 
	\hline 
	2 строки & 106 \\ 
	\hline 
	7 вспомогательных переменных(int) & 196 \\ 
	\hline 
	2 счетчика (int) & 56 \\ 
	\hline 
	\textbf{Всего} & \textbf{8064454}\\
	\hline
	\end{tabular}
	}	

	\begin{center}
  	Таблица 10. Память, потребляемая структурами в рекурсивном алгоритме нахождения расстояния Дамерау-Левенштейна\\
	\end{center}
	\scalebox{0.87}{	
	\begin{tabular}{|c|c|}
	\hline 
	Структура данных & Занимаемая память (байты) \\ 
	\hline 
	4 вспомогательных переменных (int) & 140 * 2000(максимальная глубина вызовов) = 280000\\ 
	\hline 
	2 строки & 2098 * 1000/2(Усредненное значение) = 1049000 \\ 
	\hline 
	\textbf{Всего} & \textbf{1329000}\\
	\hline	
	\end{tabular}
	}		

	
	\begin{flushleft}
	По таблицам 5,6,8 и 9 мы видим, что и при длине слов 4, и при длине слов 1000 итеративные реализации алгоритмов нахождения расстояния Левенштейна и Дамерау-Левенштейна сравнимы по потребляемой памяти, причем разница между ними всегда 28 байт. В свою очередь, рекурсивная реализация Дамерау-Левенштейна при 4 потребляет больше памяти, а при 1000 - меньше, что видно в таблицах 6, 7, 9 и 10.
	\end{flushleft}
  \begin{center}

   \subsection{Вывод}
    \end{center}  
    
	В технологической части была предоставлены реализации алгоритмов нахождения расстояния Левенштейна в итеративной форме, а также нахождения расстояния Дамерау-Левенштейна в итеративной и рекурсивной формах. Помимо этого, были предоставлены результаты тестов на правильность данных реализаций, сравнительный анализ потребляемой памяти всех реализаций на слов длиной 4 и 1000. Было выявлено, что и при длине слов 4, и при длине слов 1000 итеративные реализации алгоритмов нахождения расстояния Левенштейна и Дамерау-Левенштейна сравнимы по потребляемой памяти, причем разница между ними всегда 28 байт(3 и 0.0001 процента при длине слова 4 и 1000 соответственно). Рекурсивная реализация же при длине слов 4 потребляет на 37 процентов больше памяти, в то время как при длине слов 1000 она потребляет меньше памяти на 84 процента. По зависимостям потребляемой памяти от длин строк(рис. 5) мы видим, что рекурсивная реализация более затратна на промежутке от 1 до 59, что вызвано тем, что в функцию всегда передается копия строки, из-за того, что в python нет указателей, чтобы передавать их в функции вместо копий, иначе бы рекурсивная реализация была всегда эффективнее по памяти.	 

    \begin{center}
    	\newpage
        \section{Экспериментальная часть}
		\subsection{Постановка эксперимента}
		\begin{flushleft}
		\parindent=1cm
		Должны быть произведены замеры времени работы каждого из алгоритмов при длинах строк от 2 до 7. Каждый тест должен быть был проведен 100 раз на 100 случайных строках. Таким образом тест для одной длины строки проводится 10000 раз. Как результат должно быть взято среднее значение для уменьшения роли случайных факторов в итоге.\\
		\end{flushleft}		
		\subsection{Результаты эксперимента}		
		\begin{flushleft}
		Ниже приведены графики со сравнением требуемого процессорного времени на итеративные алгоритмы поиска расстояния Левенштейна и Дамерау-Левенштейна(рис 6) с длитнами строк 2-7 и со сравнением времени, затрачиваемого на поиск расстояния Дамерау-Левенштейна в итеративном и рекурсивном вариантах.	lev tab - поиск расстояния  Левенштейна в итеративной форме,		dam lev tab - поиск расстояния Дамерау-Левенштейна в итеративной форме,
		dam lev rec - поиск расстояния Дамерау-Левенштейна в рекурсивной форме,
		на оси абсцисс откладывается длина строк,на оси ординат - затрачиваемое время в секундах. 
				\end{flushleft}
\begin{tikzpicture}
\begin{axis}[
    	axis lines = left,
	legend pos=north west,
	ymajorgrids=true
] 
\addplot[color=red, mark=square] table[x index=0, y index=1] {lev_tab.dat};
\addplot[color=blue, mark=square] table[x index=0, y index=1] {dam_lev_tab.dat};

\addlegendentry{lev tab}
\addlegendentry{dam lev tab}
\end{axis}
\end{tikzpicture}
\begin{center}
Pис. 6: сравнение времени на поиск Левенштейна и Дамерау-Левенштейна в итеративной форме
\end{center}
\begin{tikzpicture}

\begin{axis}[
    	axis lines = left,
	legend pos=north west,
	ymajorgrids=true
] 
\addplot[color=green, mark=square] table[x index=0, y index=1] {dam_lev_rec.dat};
\addplot[color=blue, mark=square] table[x index=0, y index=1] {dam_lev_tab.dat};

\addlegendentry{dam lev rec}
\addlegendentry{dam lev tab}
\end{axis}
\end{tikzpicture}

\begin{center}
Pис. 7: сравнение времени на поиск Дамерау-Левенштейна в итеративной и рекурсивной формах
\end{center}
\begin{center}
        
      \begin{center}
      \newpage
  	Таблица 11. Время, затрачиваемое различными алгоритмами на обработку строк длин от 2 до 9(в секундах).
	\end{center}
	    \begin{tabular}{ | r | r | r | r | }
        \hline
Длина & Левенштейна & м. Дамерау-Левенштейна & р. Дамерау-Левенштейна \\ \hline
2 & 1.24e-05 & 1.43e-05 & 1.83e-05\\
3 & 1.86e-05 & 2.29e-05 & 9.10e-05\\
4 & 3.22e-05 & 3.37e-05 & 4.95е-04\\
5 & 4.20e-05 & 4.94e-05 & 2.75е-03\\
6 & 6.38e-05 & 7.42e-05 & 1.60e-02\\
7 & 8.64e-05 & 1.01e-04 & 8.83е-02\\
\hline
        \end{tabular}

\end{center}
	\begin{flushleft}
	\begin{center}
	\subsection{Вывод}
	\end{center}	
	По первому графику видно, что временные затраты на итеративный алгоритмы Левенштейна и Дамерау-Левенштейна сравнимы, но при этом алгоритм Дамерау-Левенштейна всегда медленнее. Из второго графика мы замечаем то, что рекурсивный алгоритм Дамерау-Левенштейна на порядки более затратный по времени, чем итеративный, начиная с длины строки в 5.
	\end{flushleft}
        
    \end{center}

    \begin{center}
    	\newpage
        \section*{Заключение}
        \addcontentsline{toc}{section}{Заключение}
        \begin{flushleft}
        В ходе данной лабороторной работы мною были реализовани алгоритмы Левенштейна в матричной форме и Дамерау-Левенштейна в матричной и рекурсивной форме. В ходе проверки на временные затраты было выявлено, что матричные реализации алгоритма Левенштейна и Дамерау-Левенштейна сравнимы по затрачиваемым процессорным ресурсам при длинах слов от 2 до 9(разница между ними растет от 11 до 25 процентов в пользу Левенштейна). По используемой памяти разница меньше: от 0 до 3 процентов. Также было выявлено, что начиная со строк длиной в 3 символа рекурсивный вариант Дамерау-Левенштейна на порядки более затратный по процессорному времени, чем матричный(от 12.6 до 1.3x$10^4$ раз). Это вызвано тем, что в рекурсивном виде алгоритм одни и те же расчеты производит по несколько раз, так как расстояние между оними и теми же промежуточными словами может быть запрошено в нескольких независимо вызванных функциях, в то время как в матричном варианте все промежуточные расчеты записываются в матрицу и не пересчитываются. Говоря о памяти: рекурсивная форма нахождения расстояния Дамерау-Левенштейна при длине строки в 1000 на 84 процента меньше памяти, хотя при длине строки 4 потребляет на 34 процента больше памяти. Рекурсивная форма потребляет больше памяти на интервале от 1 до 59, при всех остальных значениях длин строк итеративная реализация более затратна по памяти.
        \end{flushleft}
    \end{center}

    \begin{center}
    	\newpage
        \section*{Список литературы}
        \addcontentsline{toc}{section}{Список литературы}
        \begin{flushleft}
        [1] Дж. Макконнелл. Анализ алгоритмов. Активный обучающий подход.-М.:Техносфера, 2009.\\
        
[2] Нечёткий поиск в тексте и словаре // [Электронный ресурс]. Режим доступа: https://habr.com/ru/post/114997/ (дата обращения: 2.10.19).\\

[3] Нечеткий поиск, расстояние левенштейна алгоритм // [Электронный ресурс]. Режим доступа: https://steptosleep.ru/antananarivo-106/ (дата обращения: 2.10.19).
        \end{flushleft}
    \end{center}        
\end{document}
