\documentclass[a4paper, 14pt]{article}
\usepackage[T1,T2A]{fontenc}
\usepackage[utf8x]{inputenc}
\usepackage[english,russian]{babel}
\usepackage{listings}
\usepackage[russian]{babel}
\usepackage{graphicx}
\usepackage{listings}
\usepackage{color}
\usepackage{amsmath}
\usepackage{pgfplots}
\usepackage{url}
\usepackage{tikz}
\usepackage{enumitem}
\usetikzlibrary{calc,trees,positioning,arrows,chains,shapes.geometric,%
    decorations.pathreplacing,decorations.pathmorphing,shapes,%
    matrix,shapes.symbols}

\usepackage{titlesec}
\titleformat*{\section}{\LARGE\bfseries\centering}
\titleformat*{\subsection}{\Large\bfseries\centering}
\titleformat*{\subsubsection}{\large\bfseries}
\titleformat*{\paragraph}{\large\bfseries}
\titleformat*{\subparagraph}{\large\bfseries}
\lstset{ %
inputencoding=utf8x, extendedchars=\true,
language=python,                 % выбор языка для подсветки (здесь это С)
basicstyle=\small\sffamily, % размер и начертание шрифта для подсветки кода
numbers=left,               % где поставить нумерацию строк (слева\справа)
numberstyle=\tiny,           % размер шрифта для номеров строк
stepnumber=1,                   % размер шага между двумя номерами строк
numbersep=5pt,                % как далеко отстоят номера строк от подсвечиваемого кода
showspaces=false,            % показывать или нет пробелы специальными отступами
showstringspaces=false,      % показывать или нет пробелы в строках
showtabs=false,             % показывать или нет табуляцию в строках
frame=single,              % рисовать рамку вокруг кода
tabsize=2,                 % размер табуляции по умолчанию равен 2 пробелам
captionpos=t,              % позиция заголовка вверху [t] или внизу [b] 
breaklines=true,           % автоматически переносить строки (да\нет)
breakatwhitespace=false, % переносить строки только если есть пробел
escapeinside={\#*}{*)}   % если нужно добавить комментарии в коде
}

\usepackage{geometry}
\geometry{left=2cm}
\geometry{right=1.5cm}
\geometry{top=1cm}
\geometry{bottom=2cm}

\usepackage{pgfplots}
\usepackage{indentfirst}
\usepackage{filecontents}
\usetikzlibrary{datavisualization}
\usetikzlibrary{datavisualization.formats.functions}
\usepackage{graphicx}

\begin{document}

    \begin{titlepage}

        \begin{center}
            \large
            Государственное образовательное учреждение высшего профессионального образования\\
            “Московский государственный технический университет имени Н.Э.Баумана”
            \vspace{3cm}
            
            \textsc{Дисциплина: Анализ алгоритмов}
            \vspace{0.5cm}
                
            \textsc{Рубежный контроль 2}
            \vspace{1.5cm}
            
            {\LARGE Нахождение подстроки в строке с помощью регулярных выражений и конечного автомата.\\}
            \vspace{1.5cm}
            Студент группы ИУ7-55,\\   
            Аминов Тимур Саидович
            \vfill
            
            2019 г.
            
            \end{center}

    \end{titlepage}
    \setcounter{page}{2}

	
	\newpage       

        \label{sec:intro}

    	\newpage
        \section{Аналитическая часть}
		\parindent=1cm
		
		В рамках раздела будет дано аналитическое описание регулярных выражений и конечного автомата.
		

        \subsection{Описание алгоритмов}
        	\subsubsection{Регулярные выражения}
        
		\textbf{Регулярные выражения} – формальный язык поиска и осуществления манипуляций с подстроками в тексте, основанный на использовании метасимволов. Для поиска используется строка-образец, состоящая из символов и метасимволов и задающая правило поиска. Для манипуляций с текстом дополнительно задаётся строка замены, которая также может содержать в себе специальные символы.\\
			\subsubsection{Конечный автомат}
			
		\textbf{Конечный автомат} можно охарактеризовать множеством состояний (вершин) и переходов (дуг, соединяющих вершины). Среди состояний есть два специальных - состояние начала и конца. Если строка читается данным автоматом, то после прохода по строке, автомат должен оказаться в одном из заключительных состояний. На этом основывается алгоритм поиска подстроки в строке с помощью конечного автомата.\\
		
		
        \section{Конструкторская часть}
        
		В данной части будут рассмотрены схемы разработанных автоматов. \\
        \subsection{Разработка алгоритмов}
        На рисунках 1-4 приведены схемы автоматов поиска даты в русском, английском и текстовом форматах.\\
    	\begin{figure}[h]
        	\center{\includegraphics[scale = 0.43]{ru_date.png}}
        	\caption{Автомат для поиска даты в русском формате}
        	\label{fig:schema_standart}
        \end{figure}
        %\newpage
    	\begin{figure}[h]
        	\center{\includegraphics[scale = 0.43]{us_date.png}}
        	\caption{Автомат для поиска даты в английском формате}
        	\label{fig:schema_vinograd}
        \end{figure}
        \newpage    
    	\begin{figure}[!ht]
        	\center{\includegraphics[scale = 0.43]{text_date.png}}
        	\caption{Автомат для поиска даты в текстовом формате}
        	\label{fig:schema_vinograd_optimized}
        \end{figure}
        \newpage
        \begin{figure}[!ht]
        	\center{\includegraphics[scale = 0.43]{text_date_2.png}}
        	\caption{Автомат для поиска даты в текстовом формате (продолжение)}
        	\label{fig:schema_vinograd_optimized}
        \end{figure}
        \newpage     

        \section{Технологическая часть}
        
        В рамках этого раздела будут описаны инструментарии разработки, выбор среды, требования к ПО. Также будут предоставлены листинги конкретных реализаций алгоритмов.\\
		\subsection{Средства реализации}
		Программа была написана на языке программирования python 3.8 в среде разработки PyCharm. Я выбрал данный язык, так как уже имею определнный опыт работы с ним.\\
		\subsection{Требования к программному обеспечению}
		На вход программа должна получать текст, на выход отображать все найденные даты в формате dd.mm.yyyy, mm.dd.yyyy, а также даты, в которых название месяца записано целым или сокращенным словом.
		
        \subsection{Листинг кода}
        В листингах 1-5 представлена реализация поиска даты с помощью регулярных выражений и конечных автоматов. На листинге 1 представлен код регулярных выражений для даты в русском, английском и текстовом формате соответственно. На листинге 2 представлен класс и функции для работы с автоматом. На листинге 3 представлены инструкции автомата для поиска даты в русском формате. На листинге 4 представлены инструкции автомата для поиска даты в английском формате. На листинге 5 представлены инструкции автомата для поиска даты в текстовом формате.
		\begin{lstlisting}[label=some-code,caption=Листинг разработанных регулярных выражений]
		months = ['января', 'янв',
              'февраля', 'фев',
              'марта', 'мар',
              'апреля', 'апр',
              'мая',
              'июня', 'июн',
              'июля', 'июл',
              'августа', 'авг',
              'сентября', 'сен',
              'октября', 'окт',
              'ноября', 'ноя',
              'декабря', 'дек']

    months_str = get_months_str(months)

    res_re_ru_date = re.findall(r'(0[1-9]|[12][0-9]|3[01])[-/.](0[1-9]|1[0-2])[-/.](1\d\d\d|2[01]\d\d)', text)
    res_re_us_date = re.findall(r'(0[1-9]|1[0-2])[-/.](0[1-9]|[12][0-9]|3[01])[-/.](1\d\d\d|2[01]\d\d)', text)
    res_re_text_date = re.findall(r'([1-9]|[12][0-9]|3[01]) (' + months_str + r') (1\d\d\d|2[01]\d\d)', text)
		\end{lstlisting}
		\begin{lstlisting}[label=some-code,caption=Класс и функции для работы с автоматом]
		# Класс инструкции автомата
class Transition:
    def __init__(self, state_1, state_2, symbols, sym_ins, direction):
        self.state_1 = state_1
        self.state_2 = state_2
        self.symbols = symbols
        self.sym_ins = sym_ins
        self.direction = direction
        
    # Возвращает массив пар индексов подходящих условию подстрок
def run_automat(instructions, text, final_state):
    pos = 0
    end = len(text)
    state = 1
    start_pos = 0
    found = []

    while pos != end:
        # print(state, text[pos])
        tran = find_transition(state, text[pos], instructions)

        if tran.state_1 == 1 and tran.state_2 != 1:
            start_pos = pos

        if tran.state_2 == final_state:
            found.append((start_pos, pos + 1))

        if tran.direction == 'r':
            pos += 1
        elif tran.direction == 'l':
            pos -= 1

        state = tran.state_2

    return found
    
    # Возвращает нужную инструкцию из массива инструкций
def find_transition(state, sym, instructions):
    for i in instructions:
        if i.state_1 == state:
            if i.sym_ins == 'in':
                if sym in i.symbols:
                    return i
            elif i.sym_ins == 'not':
                if sym not in i.symbols:
                    return i
    return None
		\end{lstlisting}
		\newpage
		\begin{lstlisting}[label=some-code,caption=Автомат для поиска даты в формате dd.mm.yyyy]
		# Автомат для поиска даты в формате dd.mm.yyyy
def date_ru_automat(text):
    instructions = [Transition(1, 2, ['0'], 'in', 'r'),  # Первая цифра дня
                    Transition(1, 3, ['1', '2'], 'in', 'r'),
                    Transition(1, 4, ['3'], 'in', 'r'),
                    Transition(1, 1, ['0', '1', '2', '3'], 'not', 'r'),

                    Transition(2, 5, ['1', '2', '3', '4', '5', '6', '7', '8', '9'], 'in', 'r'),  # Вторая цифра дня
                    Transition(3, 5, ['0', '1', '2', '3', '4', '5', '6', '7', '8', '9'], 'in', 'r'),
                    Transition(4, 5, ['0', '1'], 'in', 'r'),
                    Transition(2, 1, ['1', '2', '3', '4', '5', '6', '7', '8', '9'], 'not', 'r'),
                    Transition(3, 1, ['0', '1', '2', '3', '4', '5', '6', '7', '8', '9'], 'not', 'r'),
                    Transition(4, 1, ['0', '1'], 'not', 'r'),

                    Transition(5, 6, ['.', '-', '/'], 'in', 'r'),  # Разделитель
                    Transition(5, 1, ['.', '-', '/'], 'not', 'r'),

                    Transition(6, 7, ['0'], 'in', 'r'),  # Первая цифра месяца
                    Transition(6, 8, ['1'], 'in', 'r'),
                    Transition(6, 1, ['0', '1'], 'not', 'r'),

                    Transition(7, 9, ['1', '2', '3', '4', '5', '6', '7', '8', '9'], 'in', 'r'),  # Вторая цифра месяца
                    Transition(8, 9, ['0', '1', '2'], 'in', 'r'),
                    Transition(7, 1, ['1', '2', '3', '4', '5', '6', '7', '8', '9'], 'not', 'r'),
                    Transition(8, 1, ['0', '1', '2'], 'not', 'r'),

                    Transition(9, 10, ['.', '-', '/'], 'in', 'r'),  # Разделитель
                    Transition(9, 1, ['.', '-', '/'], 'not', 'r'),

                    Transition(10, 11, ['1'], 'in', 'r'),  # Первая цифра года
                    Transition(10, 12, ['2'], 'in', 'r'),
                    Transition(10, 1, ['1', '2'], 'not', 'r'),

                    Transition(11, 13, ['0', '1', '2', '3', '4', '5', '6', '7', '8', '9'], 'in', 'r'),  # 1000 - 1999
                    Transition(13, 14, ['0', '1', '2', '3', '4', '5', '6', '7', '8', '9'], 'in', 'r'),
                    Transition(14, 100, ['0', '1', '2', '3', '4', '5', '6', '7', '8', '9'], 'in', 'r'),
                    Transition(11, 1, ['0', '1', '2', '3', '4', '5', '6', '7', '8', '9'], 'not', 'r'),
                    Transition(13, 1, ['0', '1', '2', '3', '4', '5', '6', '7', '8', '9'], 'not', 'r'),
                    Transition(14, 1, ['0', '1', '2', '3', '4', '5', '6', '7', '8', '9'], 'not', 'r'),

                    Transition(12, 15, ['0', '1'], 'in', 'r'),  # 2000 - 2199
                    Transition(15, 16, ['0', '1', '2', '3', '4', '5', '6', '7', '8', '9'], 'in', 'r'),
                    Transition(16, 100, ['0', '1', '2', '3', '4', '5', '6', '7', '8', '9'], 'in', 'r'),
                    Transition(12, 1, ['0', '1'], 'not', 'r'),
                    Transition(15, 1, ['0', '1', '2', '3', '4', '5', '6', '7', '8', '9'], 'not', 'r'),
                    Transition(16, 1, ['0', '1', '2', '3', '4', '5', '6', '7', '8', '9'], 'not', 'r'),

                    Transition(100, 1, [], 'not', 'r'),  # Продолжаем поиск
    ]

    final_state = 100
    found = run_automat(instructions, text, final_state)
    res = []

    for couple in found:
        res.append(text[couple[0]:couple[1]])

    return res
		\end{lstlisting}
		\begin{lstlisting}[label=some-code,caption=Автомат для поиска даты в формате mm.dd.yyyy]
		# Автомат для поиска даты в формате mm.dd.yyyy
def date_us_automat(text):
    instructions = [
                    Transition(1, 2, ['0'], 'in', 'r'),  # Первая цифра месяца
                    Transition(1, 3, ['1'], 'in', 'r'),
                    Transition(1, 1, ['0', '1'], 'not', 'r'),

                    Transition(2, 4, ['1', '2', '3', '4', '5', '6', '7', '8', '9'], 'in', 'r'),  # Вторая цифра месяца
                    Transition(3, 4, ['0', '1', '2'], 'in', 'r'),
                    Transition(2, 1, ['1', '2', '3', '4', '5', '6', '7', '8', '9'], 'not', 'r'),
                    Transition(3, 1, ['0', '1', '2'], 'not', 'r'),

                    Transition(4, 5, ['.', '-', '/'], 'in', 'r'),  # Разделитель
                    Transition(4, 1, ['.', '-', '/'], 'not', 'r'),

                    Transition(5, 6, ['0'], 'in', 'r'),  # Первая цифра дня
                    Transition(5, 7, ['1', '2'], 'in', 'r'),
                    Transition(5, 8, ['3'], 'in', 'r'),
                    Transition(5, 1, ['0', '1', '2', '3'], 'not', 'r'),

                    Transition(6, 9, ['1', '2', '3', '4', '5', '6', '7', '8', '9'], 'in', 'r'),  # Вторая цифра дня
                    Transition(7, 9, ['0', '1', '2', '3', '4', '5', '6', '7', '8', '9'], 'in', 'r'),
                    Transition(8, 9, ['0', '1'], 'in', 'r'),
                    Transition(6, 1, ['1', '2', '3', '4', '5', '6', '7', '8', '9'], 'not', 'r'),
                    Transition(7, 1, ['0', '1', '2', '3', '4', '5', '6', '7', '8', '9'], 'not', 'r'),
                    Transition(8, 1, ['0', '1'], 'not', 'r'),

                    Transition(9, 10, ['.', '-', '/'], 'in', 'r'),  # Разделитель
                    Transition(9, 1, ['.', '-', '/'], 'not', 'r'),

                    Transition(10, 11, ['1'], 'in', 'r'),  # Первая цифра года
                    Transition(10, 12, ['2'], 'in', 'r'),
                    Transition(10, 1, ['1', '2'], 'not', 'r'),

                    Transition(11, 13, ['0', '1', '2', '3', '4', '5', '6', '7', '8', '9'], 'in', 'r'),  # 1000 - 1999
                    Transition(13, 14, ['0', '1', '2', '3', '4', '5', '6', '7', '8', '9'], 'in', 'r'),
                    Transition(14, 100, ['0', '1', '2', '3', '4', '5', '6', '7', '8', '9'], 'in', 'r'),
                    Transition(11, 1, ['0', '1', '2', '3', '4', '5', '6', '7', '8', '9'], 'not', 'r'),
                    Transition(13, 1, ['0', '1', '2', '3', '4', '5', '6', '7', '8', '9'], 'not', 'r'),
                    Transition(14, 1, ['0', '1', '2', '3', '4', '5', '6', '7', '8', '9'], 'not', 'r'),

                    Transition(12, 15, ['0', '1'], 'in', 'r'),  # 2000 - 2199
                    Transition(15, 16, ['0', '1', '2', '3', '4', '5', '6', '7', '8', '9'], 'in', 'r'),
                    Transition(16, 100, ['0', '1', '2', '3', '4', '5', '6', '7', '8', '9'], 'in', 'r'),
                    Transition(12, 1, ['0', '1'], 'not', 'r'),
                    Transition(15, 1, ['0', '1', '2', '3', '4', '5', '6', '7', '8', '9'], 'not', 'r'),
                    Transition(16, 1, ['0', '1', '2', '3', '4', '5', '6', '7', '8', '9'], 'not', 'r'),

                    Transition(100, 1, [], 'not', 'r')  # Продолжаем поиск
    ]

    final_state = 100
    found = run_automat(instructions, text, final_state)
    res = []

    for couple in found:
        res.append(text[couple[0]:couple[1]])

    return res
		\end{lstlisting}
		\begin{lstlisting}[label=some-code,caption=Автомат для поиска даты в текстовом формате]
		# Автомат для поиска даты в текстовом формате
		def date_text_automat(text):
    a = 5
    b = 200
    c = 300

    instructions = [
        Transition(1, 2, ['1', '2'], 'in', 'r'),
        Transition(1, 3, ['3'], 'in', 'r'),
        Transition(1, 4, ['4', '5', '6', '7', '8', '9'], 'in', 'r'),
        Transition(2, 4, ['0', '1', '2', '3', '4', '5', '6', '7', '8', '9'], 'in', 'r'),
        Transition(2, a, [' '], 'in', 'r'),
        Transition(3, 4, ['0', '1'], 'in', 'r'),
        Transition(3, a, [' '], 'in', 'r'),
        Transition(4, a, [' '], 'in', 'r'),

        Transition(1, 1, ['1', '2', '3', '4', '5', '6', '7', '8', '9'], 'not', 'r'),
        Transition(2, 1, ['0', '1', '2', '3', '4', '5', '6', '7', '8', '9', ' '], 'not', 'r'),
        Transition(3, 1, ['0', '1', ' '], 'not', 'r'),
        Transition(4, 1, [' '], 'not', 'r'),

        # января, янв
        Transition(a, 6, ['я'], 'in', 'r'),
        Transition(6, 7, ['н'], 'in', 'r'),
        Transition(7, 8, ['в'], 'in', 'r'),
        Transition(8, 9, ['а'], 'in', 'r'),
        Transition(9, 10, ['р'], 'in', 'r'),
        Transition(10, 11, ['я'], 'in', 'r'),
        Transition(11, b, [' '], 'in', 'r'),
        Transition(8, b, [' '], 'in', 'r'),

        Transition(a, 1, ['я', 'ф', 'м', 'а', 'и', 'с', 'о', 'н', 'д'], 'not', 'r'),
        Transition(6, 1, ['н'], 'not', 'r'),
        Transition(7, 1, ['в'], 'not', 'r'),
        Transition(8, 1, ['а', ' '], 'not', 'r'),
        Transition(9, 1, ['р'], 'not', 'r'),
        Transition(10, 1, ['я'], 'not', 'r'),
        Transition(11, 1, [' '], 'not', 'r'),

        # февраля, фев
        Transition(a, 12, ['ф'], 'in', 'r'),
        Transition(12, 13, ['е'], 'in', 'r'),
        Transition(13, 14, ['в'], 'in', 'r'),
        Transition(14, 15, ['р'], 'in', 'r'),
        Transition(15, 16, ['а'], 'in', 'r'),
        Transition(16, 17, ['л'], 'in', 'r'),
        Transition(17, 18, ['я'], 'in', 'r'),
        Transition(18, b, [' '], 'in', 'r'),
        Transition(14, b, [' '], 'in', 'r'),

        Transition(12, 1, ['е'], 'not', 'r'),
        Transition(13, 1, ['в'], 'not', 'r'),
        Transition(14, 1, ['р', ' '], 'not', 'r'),
        Transition(15, 1, ['а'], 'not', 'r'),
        Transition(16, 1, ['л'], 'not', 'r'),
        Transition(17, 1, ['я'], 'not', 'r'),
        Transition(18, 1, [' '], 'not', 'r'),

        # марта, мар
        Transition(a, 19, ['м'], 'in', 'r'),
        Transition(19, 20, ['а'], 'in', 'r'),
        Transition(20, 21, ['р'], 'in', 'r'),
        Transition(21, 22, ['т'], 'in', 'r'),
        Transition(22, 23, ['а'], 'in', 'r'),
        Transition(23, b, [' '], 'in', 'r'),
        Transition(21, b, [' '], 'in', 'r'),

        Transition(19, 1, ['а'], 'not', 'r'),
        Transition(20, 1, ['р', 'я'], 'not', 'r'),
        Transition(21, 1, ['т', ' '], 'not', 'r'),
        Transition(22, 1, ['а'], 'not', 'r'),
        Transition(23, 1, [' '], 'not', 'r'),

        # апреля, апр
        Transition(a, 24, ['а'], 'in', 'r'),
        Transition(24, 25, ['п'], 'in', 'r'),
        Transition(25, 26, ['р'], 'in', 'r'),
        Transition(26, 27, ['е'], 'in', 'r'),
        Transition(27, 28, ['л'], 'in', 'r'),
        Transition(28, 29, ['я'], 'in', 'r'),
        Transition(29, b, [' '], 'in', 'r'),
        Transition(26, b, [' '], 'in', 'r'),

        Transition(24, 1, ['п', 'в'], 'not', 'r'),
        Transition(25, 1, ['р'], 'not', 'r'),
        Transition(26, 1, ['е', ' '], 'not', 'r'),
        Transition(27, 1, ['л'], 'not', 'r'),
        Transition(28, 1, ['я'], 'not', 'r'),
        Transition(29, 1, [' '], 'not', 'r'),

        # мая
        # Transition(,, ['м'], 'in', 'r'),
        # Transition(,, ['а'], 'in', 'r'),
        Transition(20, 30, ['я'], 'in', 'r'),
        Transition(30, b, [' '], 'in', 'r'),

        Transition(30, 1, [' '], 'not', 'r'),

        # июня, июн
        Transition(a, 31, ['и'], 'in', 'r'),
        Transition(31, 32, ['ю'], 'in', 'r'),
        Transition(32, 33, ['н'], 'in', 'r'),
        Transition(33, 34, ['я'], 'in', 'r'),
        Transition(34, b, [' '], 'in', 'r'),
        Transition(33, b, [' '], 'in', 'r'),

        Transition(31, 1, ['ю'], 'not', 'r'),
        Transition(32, 1, ['н', 'л'], 'not', 'r'),
        Transition(33, 1, ['я', ' '], 'not', 'r'),
        Transition(34, 1, [' '], 'not', 'r'),

        # июля, июл
        # Transition(, , ['и'], 'in', 'r'),
        # Transition(, , ['ю'], 'in', 'r'),
        Transition(32, 35, ['л'], 'in', 'r'),
        Transition(35, 36, ['я'], 'in', 'r'),
        Transition(36, b, [' '], 'in', 'r'),
        Transition(35, b, [' '], 'in', 'r'),

        Transition(35, 1, ['я', ' '], 'not', 'r'),
        Transition(36, 1, [' '], 'not', 'r'),

        # августа, авг
        # Transition(,, ['а'], 'in', 'r'),
        Transition(24, 37, ['в'], 'in', 'r'),
        Transition(37, 38, ['г'], 'in', 'r'),
        Transition(38, 39, ['у'], 'in', 'r'),
        Transition(39, 40, ['с'], 'in', 'r'),
        Transition(40, 41, ['т'], 'in', 'r'),
        Transition(41, 42, ['а'], 'in', 'r'),
        Transition(42, b, [' '], 'in', 'r'),
        Transition(38, b, [' '], 'in', 'r'),

        Transition(37, 1, ['г'], 'not', 'r'),
        Transition(38, 1, ['у', ' '], 'not', 'r'),
        Transition(39, 1, ['с'], 'not', 'r'),
        Transition(40, 1, ['т'], 'not', 'r'),
        Transition(41, 1, ['а'], 'not', 'r'),
        Transition(42, 1, [' '], 'in', 'r'),

        # сентября, сен // 43 - отсутствует
        Transition(a, 44, ['с'], 'in', 'r'),
        Transition(44, 45, ['е'], 'in', 'r'),
        Transition(45, 46, ['н'], 'in', 'r'),
        Transition(46, 47, ['т'], 'in', 'r'),
        Transition(47, 48, ['я'], 'in', 'r'),
        Transition(48, 49, ['б'], 'in', 'r'),
        Transition(49, 50, ['р'], 'in', 'r'),
        Transition(50, 51, ['я'], 'in', 'r'),
        Transition(51, b, [' '], 'in', 'r'),
        Transition(46, b, [' '], 'in', 'r'),

        Transition(44, 1, ['е'], 'not', 'r'),
        Transition(45, 1, ['н'], 'not', 'r'),
        Transition(46, 1, ['т', ' '], 'not', 'r'),
        Transition(47, 1, ['я'], 'not', 'r'),
        Transition(48, 1, ['б'], 'not', 'r'),
        Transition(49, 1, ['р'], 'not', 'r'),
        Transition(50, 1, ['я'], 'not', 'r'),
        Transition(51, 1, [' '], 'not', 'r'),

        # октября, окт // 52 - отсутствует
        Transition(a, 53, ['о'], 'in', 'r'),
        Transition(53, 54, ['к'], 'in', 'r'),
        Transition(54, 55, ['т'], 'in', 'r'),
        Transition(55, 56, ['я'], 'in', 'r'),
        Transition(56, 57, ['б'], 'in', 'r'),
        Transition(57, 58, ['р'], 'in', 'r'),
        Transition(58, 59, ['я'], 'in', 'r'),
        Transition(59, b, [' '], 'in', 'r'),
        Transition(55, b, [' '], 'in', 'r'),

        Transition(53, 1, ['к'], 'not', 'r'),
        Transition(54, 1, ['т'], 'not', 'r'),
        Transition(55, 1, ['я', ' '], 'not', 'r'),
        Transition(56, 1, ['б'], 'not', 'r'),
        Transition(57, 1, ['р'], 'not', 'r'),
        Transition(58, 1, ['я'], 'not', 'r'),
        Transition(59, 1, [' '], 'not', 'r'),

        # ноября, ноя // 60 - отсутствует
        Transition(a, 61, ['н'], 'in', 'r'),
        Transition(61, 62, ['о'], 'in', 'r'),
        Transition(62, 63, ['я'], 'in', 'r'),
        Transition(63, 64, ['б'], 'in', 'r'),
        Transition(64, 65, ['р'], 'in', 'r'),
        Transition(65, 66, ['я'], 'in', 'r'),
        Transition(66, b, [' '], 'in', 'r'),
        Transition(63, b, [' '], 'in', 'r'),

        Transition(61, 1, ['о'], 'not', 'r'),
        Transition(62, 1, ['я'], 'not', 'r'),
        Transition(63, 1, ['б', ' '], 'not', 'r'),
        Transition(64, 1, ['р'], 'not', 'r'),
        Transition(65, 1, ['я'], 'not', 'r'),
        Transition(66, 1, [' '], 'not', 'r'),

        # декабря, дек // 67 - отсутствует
        Transition(a, 68, ['д'], 'in', 'r'),
        Transition(68, 69, ['е'], 'in', 'r'),
        Transition(69, 70, ['к'], 'in', 'r'),
        Transition(70, 71, ['а'], 'in', 'r'),
        Transition(71, 72, ['б'], 'in', 'r'),
        Transition(72, 73, ['р'], 'in', 'r'),
        Transition(73, 74, ['я'], 'in', 'r'),
        Transition(74, b, [' '], 'in', 'r'),
        Transition(70, b, [' '], 'in', 'r'),

        Transition(68, 1, ['е'], 'not', 'r'),
        Transition(69, 1, ['к'], 'not', 'r'),
        Transition(70, 1, ['а', ' '], 'not', 'r'),
        Transition(71, 1, ['б'], 'not', 'r'),
        Transition(72, 1, ['р'], 'not', 'r'),
        Transition(73, 1, ['я'], 'not', 'r'),
        Transition(74, 1, [' '], 'not', 'r'),

        # Год
        Transition(b, 81, ['1'], 'in', 'r'),  # Первая цифра года
        Transition(b, 82, ['2'], 'in', 'r'),
        Transition(b, 1, ['1', '2'], 'not', 'r'),

        Transition(81, 83, ['0', '1', '2', '3', '4', '5', '6', '7', '8', '9'], 'in', 'r'),  # 1000 - 1999
        Transition(83, 84, ['0', '1', '2', '3', '4', '5', '6', '7', '8', '9'], 'in', 'r'),
        Transition(84, c, ['0', '1', '2', '3', '4', '5', '6', '7', '8', '9'], 'in', 'r'),
        Transition(81, 1, ['0', '1', '2', '3', '4', '5', '6', '7', '8', '9'], 'not', 'r'),
        Transition(83, 1, ['0', '1', '2', '3', '4', '5', '6', '7', '8', '9'], 'not', 'r'),
        Transition(84, 1, ['0', '1', '2', '3', '4', '5', '6', '7', '8', '9'], 'not', 'r'),

        Transition(82, 85, ['0', '1'], 'in', 'r'),  # 2000 - 2199
        Transition(85, 86, ['0', '1', '2', '3', '4', '5', '6', '7', '8', '9'], 'in', 'r'),
        Transition(86, c, ['0', '1', '2', '3', '4', '5', '6', '7', '8', '9'], 'in', 'r'),
        Transition(82, 1, ['0', '1'], 'not', 'r'),
        Transition(85, 1, ['0', '1', '2', '3', '4', '5', '6', '7', '8', '9'], 'not', 'r'),
        Transition(86, 1, ['0', '1', '2', '3', '4', '5', '6', '7', '8', '9'], 'not', 'r'),

        Transition(c, 1, [], 'not', 'r')
    ]

    final_state = c
    found = run_automat(instructions, text, final_state)
    res = []

    for couple in found:
        res.append(text[couple[0]:couple[1]])

    return res
		\end{lstlisting}
        
    	\newpage
        \section*{Заключение}
        \addcontentsline{toc}{section}{Заключение}
        
        Выполнен рубежный контроль по нахождению подстроки в строке при помощи регулярных выражений и конечного автомата.
     
\end{document}
